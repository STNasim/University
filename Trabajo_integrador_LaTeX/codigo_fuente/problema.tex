El problema de los 100 prisioneros tiene diferentes versiones en la literatura. La siguiente versión es de \href{https://en.wikipedia.org/wiki/Philippe_Flajolet}{\textbf{Philippe Flajolet}} y \href{https://en.wikipedia.org/wiki/Robert_Sedgewick_(computer_scientist)}{\textbf{Robert Sedgewick}} \cite{flajolet2009analytic}.

\textit{``El director de una prisión ofrece una última oportunidad a 100 condenados a muerte, que están numerados del 1 al 100. Una habitación contiene un armario con 100 cajones. El director pone al azar el número de un preso en cada cajón cerrado. Los prisioneros entran en la habitación, uno tras otro. Cada preso puede abrir y mirar en 50 cajones en cualquier orden. Los cajones se cierran de nuevo después. Si, durante esta búsqueda, cada preso encuentra su número en uno de los cajones, todos los presos son indultados. Si un solo preso no encuentra su número, todos los presos mueren. Antes de que el primer preso entre en la habitación, los presos pueden discutir la estrategia, pero no pueden comunicarse una vez que el primer preso entra para mirar en los cajones. ¿Cuál es la mejor estrategia de los presos?''}

Si cada preso selecciona al azar 50 cajones , la probabilidad de que un solo preso encuentre su número es del 50\%. Por lo tanto, la probabilidad de que todos los presos encuentren sus números es el producto de las probabilidades individuales, que es (1/2)$^{100}$ $\approx$ 0.0000000000000000000000000000008, un número muy pequeño. La situación parece desesperada.