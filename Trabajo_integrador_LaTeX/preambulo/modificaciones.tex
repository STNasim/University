% \definecolor{codegreen}{rgb}{0,0.6,0}
% \definecolor{codegray}{rgb}{0.5,0.5,0.5}
% \definecolor{codepurple}{rgb}{0.58,0,0.82}
% \definecolor{backcolour}{rgb}{0.95,0.95,0.92}
% \lstdefinestyle{mystyle}{
% 	backgroundcolor=\color{backcolour},   
% 	commentstyle=\color{codegreen},
% 	keywordstyle=\color{magenta},
% 	numberstyle=\tiny\color{codegray},
% 	stringstyle=\color{codepurple},
% 	basicstyle=\ttfamily\footnotesize,
% 	breakatwhitespace=false,         	
% 	breaklines=true,                 
% 	captionpos=b,                    
% 	keepspaces=true,                 
% 	numbers=left,                    
% 	numbersep=5pt,                  
% 	showspaces=false,                
% 	showstringspaces=false,
% 	showtabs=false,                  
% 	tabsize=2
% }
% \lstset{style=mystyle}

\setlength{\headheight}{34pt}

%Encabezados y pie de página
\fancyhf{}
\fancyhead[L]{Facultad de Ingenieria UNER\\\includegraphics[width=0.1\textwidth]{imagenes/logo_fiuner.png}}
\fancyhead[R]{Curso Introducción a LaTeX} 

\fancyfoot[C]{\thepage}
\fancyfoot[L]{\leftmark}
\fancyfoot[R]{\rightmark}
\renewcommand{\footrulewidth}{0.4pt}

\pagestyle{fancy}

%Nuevos comandos para secciones sin numeración y agregados al indice
\newcommand{\seccion}[1]
{
    \section*{#1}
    \addcontentsline{toc}{section}{#1}
    \markboth{#1}{}
}
\newcommand{\subseccion}[1]
{
    \subsection*{#1}
    \addcontentsline{toc}{subsection}{#1}
    \markright{#1}
}