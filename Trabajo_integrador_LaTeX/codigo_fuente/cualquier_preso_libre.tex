En la variante en la que cualquier prisionero que encuentre su número queda libre, la probabilidad esperada de supervivencia de un individuo dada una permutación aleatoria es la siguiente:
\\\\
Sin estrategia: $\frac{1}{2}$\\
Con la estrategia original:
\begin{equation}
    \begin{split}
        [1-\ln(2)]\cdot 1+\sum_{k=[n/2]+1}^{N} \frac{1}{k}\left(1-\frac{k}{n}\right)&=1-\ln(2)+\sum_{[n/2]+1}^{N}\frac{1}{k}-\sum_{k=[n/2]+1}^{N}\frac{1}{n} \\
        &=1-\sum_{n=51}^{100} \frac{1}{n} \\
        &=1-(H_{100}-H_{50}) \\
        &\approx 0.31183
    \end{split}
\end{equation}

Es de destacar que aunque recibimos los mismos valores esperados, son de distribuciones muy diferentes. Con la segunda estrategia, algunos prisioneros simplemente están destinados a morir o vivir dada una permutación particular, y con la primera estrategia (es decir, sin estrategia), hay "verdaderamente" una probabilidad de 1/2 para cada permutación.