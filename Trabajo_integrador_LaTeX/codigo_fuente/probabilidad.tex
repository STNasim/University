En el problema inicial, los 100 prisioneros tienen éxito si el ciclo más largo de la permutación tiene una longitud de 50 como máximo. Su probabilidad de supervivencia es, por tanto, igual a la probabilidad de que una permutación aleatoria de los números del 1 al 100 no contenga ningún ciclo de longitud mayor de 50. Esta probabilidad se determina con la ecuación \ref{eq:prob_exito_general}.

\begin{equation}
   \begin{pmatrix}
    100\\ 
    l
    \end{pmatrix}
\cdot(l-1)!\cdot(100-l)!=\frac{100!}{l}
\label{eq:prob_exito_general}
\end{equation}


La probabilidad de que una permutación aleatoria ( distribuida uniformemente ) no contenga un ciclo de longitud superior a 50 se calcula con la fórmula para eventos únicos y la fórmula para eventos complementarios \ref{eq:prob_exito_comp}

\begin{equation}\label{eq:prob_exito_comp}
    \begin{split}
        1-\frac{1}{100!}\left(\frac{100!}{51}+\dots+\frac{100!}{100}\right)&=1-\left(\frac{1}{51}+\dots+\frac{1}{100}\right) \\
        &=1-\sum_{n=51}^{100} \frac{1}{n} \\
        &=1-(H_{100}-H_{50}) \\
        &\approx 0.31183
    \end{split}
\end{equation}

dónde $H_n$ es el n-ésimo número armónico. Por lo tanto, utilizando la estrategia de seguimiento del ciclo, los presos sobreviven en un sorprendente 31\% de los casos.