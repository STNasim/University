\begin{center}
    \Large
    \textbf{Problema de los 100 prisioneros}
        
    \vspace{0.5cm}
    
    \large
    {Salim Taleb Nasim Anibal}
       
    \vspace{0.7cm}
\end{center}
\begin{abstract}
El dilema de los 100 prisioneros y 100 cajones es un problema en la teoría de la probabilidad y la combinatoria. Consiste en que cada uno de 100 prisioneros debe encontrar su número en uno de los 100 cajones para sobrevivir y si alguno no lo encuentra, todos morirán; y, cada prisionero puede abrir sólo 50 cajones y no puede comunicarse con los demás prisioneros, excepto en el debate previo de la estrategia.
A primera vista, la situación es desesperada, pero existe una estrategia que ofrece a los cautivos una oportunidad de supervivencia aproximadamente del 30\%. El científico en computación danés \href{https://scholar.google.com/citations?user=0za9d6AAAAAJ&hl=es}{\textbf{Peter Bro Miltersen}} fue el primero en proponer este problema en el 2003.
\end{abstract}
